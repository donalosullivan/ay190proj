\documentclass[11pt,letterpaper]{article}

% Load some basic packages that are useful to have
% and that should be part of any LaTeX installation.
%
% be able to include figures
\usepackage{graphicx}
% get nice colors
\usepackage{xcolor}

% change default font to Palatino (looks nicer!)
\usepackage[latin1]{inputenc}
\usepackage{mathpazo}
\usepackage[T1]{fontenc}
% load some useful math symbols/fonts
\usepackage{latexsym,amsfonts,amsmath,amssymb}

% comfort package to easily set margins
\usepackage[top=1in, bottom=1in, left=1in, right=1in]{geometry}

% control some spacings
%
% spacing after a paragraph
\setlength{\parskip}{.15cm}
% indentation at the top of a new paragraph
\setlength{\parindent}{0.0cm}


\begin{document}

\begin{center}
\Large
Ay190 -- Final Project\\
Scott Barenfeld\\
Date: \today
\end{center}

\section{Introduction}
Radio interferometers allow information from multiple radio antennae to be 
combined, giving the interferometer as a whole much greater resolution than 
a single dish on its own.  Because radio telescopes measure the incoming 
electric field itself (as a voltage), instead of just its intensity, the 
amplitude and phase of the electric field is preserved.  In an array, the 
time-averaged product of the voltages measured by each pair of antennae is 
reported as a \emph{visibility}.  Between antennas $i$ and $j$, the 
visibility is defined as:

\begin{equation}\label{eq:vis}
\mathcal{V}=\int_{4\pi} \! A(\hat{r})I(\hat{r})e^{-2\pi i\nu\vec{B}\cdot\hat{r}/c} \, \mathrm{d}\Omega,
\end{equation} 

where $A$ is the beam pattern of an individual antenna (in this case, a Gaussian 
with $\sigma$=1 arcmin), $I$ is the intensity 
of the sky, $\nu$ is the frequency of the radiation, $\vec{B}$ is the baseline 
vector between $i$ and $j$ ($\vec{r_i}-\vec{r_j}$), and $\hat{r}$ is the 
direction along the line of sight.  In the limit of a narrow field of view, 
Equation \ref{eq:vis} can be rewritten using a change of variables as: 

\begin{equation}\label{eq:vis2}
\mathcal{V}(u,v)=\int_{-\infty}^{\infty} \! \int_{-\infty}^{\infty} \! \frac{A(l,m)I(l,m)}{\sqrt{1-l^2-m^2}}e^{-2\pi i(ul+vm)} \, \mathrm{d}l \, \mathrm{d}m,
\end{equation}
where $(u,v)$ is the baseline vector between the two antennae in units of 
wavelengths and $l$ and $m$ are the angles (in radians) on the sky relative 
to the line of sight.  Equation \ref{eq:vis2} is simply a two-dimensional 
Fourier transform.  Thus, an image of the intensity pattern on the sky can 
be generated from the inverse Fourier transform of the measured visibilities.  

\section{Setup}


\section{Imaging with a DFT}


\section{Imaging with an FFT}

\section{Discussion}


\end{document}

